\chapter{Musician Self-Reports}\label{chap:mp_appendix}

The following tables contain the free text responses given by participants in the online surveys completed following each condition in the experiment, as described in Section~\ref{sec:mp_experimental_procedure}. Not all participants chose to provide free text responses, and some performers only commented on a small number of conditions.

Responses have been edited to ensure the anonymity of participants, such that direct references to participant’s names have been replaced by generic statements corresponding to their instrument, instead (e.g., ‘the drummer’, ‘the pianist’ etc.). No other editorial changes have been made and the spelling and grammar of the original responses (which were entered by participants on mobile phones, tablets, and laptops) have been maintained. 

Note finally that, as participants were not informed about the addition of latency and jitter to their partner’s performance, their comments occasionally reflect a naive understanding of the procedure and thus may be misleading in light of the full methodological description found in Section~\ref{sec:mp_methods}. For instance, one participant misconstrued the change in tempo of their performance as the result of their partner hearing a metronome playing at a slower tempo through their headphones.\clearpage

\begin{sidewaystable}[]
\centering
\caption{Latency: 0 ms, Jitter: 0 ms}
\begin{tabular}{l c c p{15cm}}
\toprule
\textbf{Instrument} & \textbf{Duo} & \textbf{Session} & \textbf{Comment} \\ \midrule
Piano & 1 & 1 & Possibly just warming up, but it felt very easy to interact with my partner and respond to our behaviours \\ 
Piano & 1 & 2 & Smooth, responded to little rhythmic cues and things, felt natural \\ 
Drums & 3 & 1 & I could hear [the pianist] well and it we came up with some nice interactive phrases (call and response, repetition, accents played together) \\ 
Piano & 3 & 1 & Most successful yet - tempo was solid from the start and both played with more precision and clarity. Interaction was good \\ 
Drums & 3 & 2 & It was easy to communicate and we had some nice call and response going, my comping was a bit sloppy which was throwing me off a bit \\ 
Piano & 3 & 2 & Pretty solid tempo, some more interaction this time \\ 
Piano & 4 & 1 & Felt slightly more difficult to coordinate, and to be creative - perhaps purely a natural dip though. \\ 
Piano & 4 & 2 & Didn't seem to be any/much feed disruption. As such, very creative and interactive performance. \\ \bottomrule
\end{tabular}
\end{sidewaystable}\clearpage

\begin{sidewaystable}[]
\centering
\caption{Latency: 23 ms, Jitter: 0 ms}
\begin{tabular}{l c c p{15cm}}
\toprule
\textbf{Instrument} & \textbf{Duo} & \textbf{Session} & \textbf{Comment} \\ \midrule
Drums & 1 & 1 & Sounds better when we slow down to meet each other even when we're not in sync yet. I don't know why! Fills do not make anything easier for anyone which is a shame because they are fun to play \\ 
Piano & 1 & 1 & We were just on totally different wavelengths. I ended up ignoring [the drummer] totally, both audio and visual, just to see if he could catch on, but it didn’t make it easier \\ 
Drums & 1 & 2 & They feel like they're getting easier but that requires less concentration and interactions \\ 
Piano & 1 & 2 & This one felt good - I even forgot my brief and got a bit out of hand melodically \\ 
Drums & 3 & 1 & It was much easier to hold the tempo and the difference in quality of the visual did not matter as we could follow each other’s playing effectively. I felt relieved to be able to interact much more easily, it felt like we were really playing together \\ 
Piano & 3 & 1 & We were playing in the same tempo this time - more dynamic variation and interaction than before \\ 
Piano & 3 & 2 & Started out a bit hesitant but interaction improved as we went on \\ 
Piano & 4 & 1 & It felt even more difficult to synchronise our beat. It seemed that [the drummer] was setting a slightly slower tempo than mine, which occasionally got slightly slower. This was difficult to 'sit into' the groove, and make creative ideas. \\ 
Piano & 4 & 2 & Good creative energy and flow, particularly in second half of Condition. \\ \bottomrule
\end{tabular}
\end{sidewaystable}\clearpage

\begin{sidewaystable}[]
\centering
\caption{Latency: 23 ms, Jitter: 0.5x}
\begin{tabular}{l c c p{15cm}}
\toprule
\textbf{Instrument} & \textbf{Duo} & \textbf{Session} & \textbf{Comment} \\ \midrule
Drums & 1 & 1 & Felt like a battle of wills. When I finally gave in and slowed, keys suddenly jumped ahead as if to mess with me! \\ 
Piano & 1 & 1 & We definitely slowed down again, but this time I figured out why because what I could see and hear seemed to all match up. We just couldn’t manage to communicate time properly somehow \\ 
Drums & 1 & 2 & We both played big heavy stuff and it seemed to help us sound (not necessarily be) in time \\ 
Piano & 1 & 2 & A good final one, thank you [NB. participant refers directly to author here]. We got beyond synchronicity and got onto shape and musicality \\ 
Drums & 2 & 2 & Felt like I dragged a bit. \\ 
Drums & 3 & 1 & It was easy to communicate but it didn’t feel 100\% fluid \\ 
Piano & 3 & 1 & Tempo felt relatively secure - not as much interaction \\ 
Drums & 3 & 2 & I could hear [the pianist] well and in time and felt like we played together. We could play intricate rhythms together confidently \\ 
Piano & 3 & 2 & Tried to push the tempo more with walking bass to avoid slowing down - felt quite draggy \\ 
Piano & 4 & 1 & Relatively simple again. \\ 
Piano & 4 & 2 & Pretty well synchronised, good groove. \\ \bottomrule
\end{tabular}
\end{sidewaystable}\clearpage

\begin{sidewaystable}[]
\centering
\caption{Latency: 23 ms, Jitter: 1.0x}
\begin{tabular}{l c c p{15cm}}
\toprule
\textbf{Instrument} & \textbf{Duo} & \textbf{Session} & \textbf{Comment} \\ \midrule
Drums & 1 & 1 & That felt harder than the warm up but that could be because now I know it's the real deal \\ 
Piano & 1 & 1 & The combination of seeing and hearing meant that we could shape the performance together \\ 
Piano & 1 & 2 & Again we felt connected and interactive \\ 
Drums & 3 & 1 & I felt like it wasn’t as smooth as before but couldn’t tell why, it seemed that we were mostly hearing each other in time \\ 
Piano & 3 & 1 & Unsure whether tempo dipped naturally here or not - felt like we slowed down after the click \\ 
Drums & 3 & 2 & It was mostly easy to communicate but there were a couple of moments where we got a bit out of time \\ 
Piano & 3 & 2 & Tempo seemed to start to deviate halfway through? \\ 
Piano & 4 & 1 & We were pretty good at adjusting to the sudden changes of tempo, having become accustomed to it over the last few Conditions. \\ 
Piano & 4 & 2 & Moments of ride cymbal 'jolting' were subsumed effectively within the groove. \\ 
Piano & 5 & 2 & Fluctuation in tempo \\ \bottomrule
\end{tabular}
\end{sidewaystable}\clearpage

\begin{sidewaystable}[]
\centering
\caption{Latency: 45 ms, Jitter: 0.0x}
\begin{tabular}{l c c p{15cm}}
\toprule
\textbf{Instrument} & \textbf{Duo} & \textbf{Session} & \textbf{Comment} \\ \midrule
Drums & 1 & 1 & Really frustrating, felt as if keys were dragging slightly so tried to continue at initial tempo. Never lined up so tried to join them. Inevitably slowed down more and more. Frustrating to not be able to use body language effectively \\ 
Piano & 1 & 1 & Interesting one, I thought we were communicating quite effectively but this time I was looking at the screen more and could see that [the drummer] was finding it much more difficult than me. He was trying to dictate a different tempo but it seemed okay to me! It threw me off a bit \\ 
Drums & 1 & 2 & We weren't that out of sync we just slowed down lots! At first this was frustrating - this is now almost guaranteed and very much expected \\ 
Piano & 1 & 2 & The rallentando effect was there but less prominently so \\ 
Drums & 3 & 1 & I felt like we dragged a bit and something was a little bit off but it was still possible to mostly play together \\ 
Piano & 3 & 1 & Some dip in tempo this time at the start - less pronounced than before. Adjusted to the ride cymbal mainly - some dynamic interaction \\ 
Drums & 3 & 2 & We didn’t have to look at each other much to understand what was going on. I think there was an awareness that something might be weird so there was a certain apprehension in sparser sections \\ 
Piano & 3 & 2 & Still felt like we slow down immediately - trying to push tempo. Some dynamic changes. \\ 
Piano & 4 & 1 & We hit a creative peak in the solo! The small disruptions of beats being lost/tempo increasing or decreasing helped with the energy of the solo. \\ \bottomrule
\end{tabular}
\end{sidewaystable}\clearpage

\begin{sidewaystable}[]
\centering
\caption{Latency: 45 ms, Jitter: 0.5x}
\begin{tabular}{l c c p{15cm}}
\toprule
\textbf{Instrument} & \textbf{Duo} & \textbf{Session} & \textbf{Comment} \\ \midrule
Drums & 1 & 1 & Slowed down but found a groove. Feels more interactive when it's harder to sync up - you can really just ignore each other and do your own thing at all \\ 
Piano & 1 & 1 & I didn’t notice us slowing down but did notice that we had slowed down \\ 
Drums & 1 & 2 & Eventually settled into a groove but felt like I was wrestling for tempo control \\ 
Piano & 1 & 2 & There were a few instances of asynchronicity but I think as a result of performer error rather than communication failure \\ 
Drums & 2 & 1 & I have a tendency to push ahead when others are so far behind, but otherwise liked the 'feel' of this (feeling behind the beat rather than in a different place). \\ 
Drums & 3 & 1 & I felt like the tempo was dragging a lot in inconsistent amounts and I was finding it difficult to predict when the next bass crotchet was coming in - this made it very difficult to hold the tempo and communicate in an effective way \\ 
Piano & 3 & 1 & Initially thought [the drummer] might have just started off a bit slow from the initial click - after a chorus it was clear he had a different tempo in his ears and I decided to go with his ride cymbal \\ 
Drums & 3 & 2 & It was mostly easy to communicate but I felt like I had to play ahead of [the pianist] some of the time \\ 
Piano & 3 & 2 & Slowed down at the beginning \\ 
Piano & 4 & 1 & The tempo feed from [the drummer] immediately got slower, and got increasingly slow throughout. Cadence points became increasingly difficult to execute convincingly. Jolts of lack of synchronised tempo made creativity difficult. \\ 
Piano & 4 & 2 & Similar, pretty good interaction. Pretty creative, nothing jaw-dropping. \\ 
Piano & 5 & 2 & Felt like [the drummer] was slowing down \\ \bottomrule
\end{tabular}
\end{sidewaystable}\clearpage

\begin{sidewaystable}[]
\centering
\caption{Latency: 45 ms, Jitter: 1.0x}
\begin{tabular}{l c c p{15cm}}
\toprule
\textbf{Instrument} & \textbf{Duo} & \textbf{Session} & \textbf{Comment} \\ \midrule
Drums & 1 & 1 & Plowing on at right tempo didn't really seem like an option, esp in a performance where cohesiveness probs more important than tempo accuracy. Slowed down a lot but made it quite a fun heavy groove when we got the hang of it \\ 
Piano & 1 & 1 & The experience felt similar to the previous one but throughout rather than coming and going. We slowed down dramatically and didn’t manage to recommunicate a new tempo \\ 
Drums & 1 & 2 & Actually remained fairly in tempo until later on where we slowed down. \\ 
Piano & 1 & 2 & To me it felt like a constant rallentando that kept failing to land and was extremely frustrating \\ 
Drums & 3 & 1 & Again it was very difficult to hold the tempo consistent and I felt like we dragged a lot \\ 
Piano & 3 & 1 & More difficult to play together than previous test - felt like tempo was fluctuating more \\ 
Drums & 3 & 2 & I was playing ahead of [the pianist] a lot of the time but I felt that if I slowed down he would slow down even more so I tried to hold the tempo. We nevertheless dragged on the tempo \\ 
Piano & 3 & 2 & Initial tempo was slower than the click - some push and pull but eventually agreed \\ 
Piano & 4 & 1 & Ground to a halt! Similar issue to last time. At points I tried to be on the 'front side' of the beat to keep things hanging together, but that didn't seem to alter [the drummer]'s sense of slowing down! \\ 
Piano & 4 & 2 & A more creative and interactive performance. \\ 
Drums & 5 & 1 & Drummer dropping stick makes us concentrate more \\ \bottomrule
\end{tabular}
\end{sidewaystable}\clearpage

\begin{sidewaystable}[]
\centering
\caption{Latency: 90ms, Jitter: 0.0x}
\begin{tabular}{l c c p{15cm}}
\toprule
\textbf{Instrument} & \textbf{Duo} & \textbf{Session} & \textbf{Comment} \\ \midrule
Drums & 1 & 1 & Slowed down together. Wasn't awful but certainly a far cry from good. Getting frustrated at being slowed down. Did a wonky groove. Sounded fine \\ 
Piano & 1 & 1 & It was like playing in sludge. I think we both kind of gave up near the end and accepted that we would just play at 50 bpm \\ 
Drums & 1 & 2 & Next time I will keep tempo. And see what happens \\ 
Piano & 1 & 2 & I could tell it wasn’t going to go well from when our nods to the count in were out of sync on the screen and indeed there were points when I just ignored all feedback and tried to replicate 120 bpm at a guess \\ 
Drums & 2 & 1 & Like trying to row through a sea of mars bars. \\ 
Drums & 3 & 1 & Again we dragged a lot and I felt like we were both hearing each other at delayed times and as a result waiting for each other with the effect of slowing down. We tried to communicate visually by making eye contact but it was very difficult to do so effectively \\ 
Piano & 3 & 1 & I knew we had the same count in this time from head movement - either it changes straight away or gets slower - this one was seriously slow! We were both trying to figure it out a little more through visual contact in this test \\ 
Drums & 3 & 2 & I had to play ahead of [the pianist] to hold the tempo. This sometimes made it difficult to play phrases coherently and we dragged a lot \\ 
Piano & 3 & 2 & Not having any success trying to dictate the tempo - letting drums lead assuming he has some other tempo info in his headphones \\ 
Piano & 4 & 1 & Aside from the start of the Condition, my feed from [the drummer] seemed natural. \\ 
Piano & 4 & 2 & Not much feed disruption. Pretty interactive despite moments of jolts. \\ \bottomrule
\end{tabular}
\end{sidewaystable}\clearpage

\begin{sidewaystable}[]
\centering
\caption{Latency: 90ms, Jitter: 0.5x}
\begin{tabular}{l c c p{15cm}}
\toprule
\textbf{Instrument} & \textbf{Duo} & \textbf{Session} & \textbf{Comment} \\ \midrule
Drums & 1 & 1 & Hard to tell after a tricky one whether it's actually hard to coordinate or if we are just being bad because we think we should be finding it hard \\ 
Piano & 1 & 1 & To try and avoid the issues of the previous condition I avoided looking at the screen and focused on what I could hear - from my end the performance seemed much more successful, but less interactive given I wasn’t watching my partner \\ 
Drums & 1 & 2 & Tried to plow on in the middle. Almost worked (same tempo but different places within the beat it felt like). Was too close to not join keys which threw things out again \\ 
Piano & 1 & 2 & Similar to last time but coming and going - there were moments where I felt like we found each other again but then it would deteriorate once more \\ 
Drums & 3 & 1 & We dragged a lot and I played ahead of [the pianist] most of the time - it felt like if I slowed down he would also slow down and we would keep slowing down so I ended up just playing half a beat or so ahead of him. \\ 
Piano & 3 & 1 & Didn’t feel like we ever locked in tempo - little interaction as a result \\ 
Drums & 3 & 2 & I felt like I had to play ahead of [the pianist] but we didn’t manage to reach a consistent tempo. Nevertheless we could phrase together at a few points, but not in a clean way \\ 
Piano & 3 & 2 & I can’t figure it out! This time I tried to resist and keep the initial tempo but it didn’t work. I also tried going double time to see if he followed. \\ 
Drums & 4 & 1 & I think I worked out where I had to play in relation to [the pianist]. It seemed that it appeared to him we were playing in sync, however I displaced my best by one triplet quaver against the pulse I got from [the pianist] \\ 
Piano & 4 & 1 & Seemed to be occasional moments of triplet-quaver 'slips' from [the drummer], but these were so short that they didn't disrupt the musicality. \\ 
Piano & 5 & 1 & Becoming more accustomed to the setup helps communication \\ 
Piano & 5 & 2 & I noticed the CCTV fluctuating at the beginning of the performance \\ \bottomrule
\end{tabular}
\end{sidewaystable}\clearpage

\begin{sidewaystable}[]
\centering
\caption{Latency: 90ms, Jitter: 1.0x}
\begin{tabular}{l c c p{15cm}}
\toprule
\textbf{Instrument} & \textbf{Duo} & \textbf{Session} & \textbf{Comment} \\ \midrule
Drums & 1 & 1 & Generally vaguely together with moments of wild tempo fluctuations. Popped an extra beat in in a middle fill which ended up working quite nicely - it's fun when stuff works even when really it's not working very well \\ 
Piano & 1 & 1 & When things were going awry I tried to just forge ahead and it seemed to fix itself, at least based on what I could hear. When I looked at the screen it was more difficult to just forge ahead \\ 
Drums & 1 & 2 & All over the place. Wild fluctuations in tempo ensued. Not very together. I went for the match keys approach not the plow on approach. Maybe that was wrong \\ 
Piano & 1 & 2 & Absolute carnage and I think it sounded utterly awful \\ 
Drums & 2 & 1 & When the time isn't settled I have a tendency to tense up - you would hear the negative effect on the sound of a real ride cymbal. \\ 
Drums & 2 & 2 & Lost the form. \\ 
Drums & 3 & 1 & It took a while to figure out how to play together but just before the end we managed to begin to play in time with each other. Eye contact and watching the fingers on the keys helped. Smiling at each other seemed to indicate that we both knew something was off \\ 
Piano & 3 & 1 & Least successful performance - didn’t seem like we ever agreed on the tempo even though we were trying to communicate. Ride cymbal is the most obvious indicator of tempo but I still struggled to hear the beat clearly \\ 
Drums & 3 & 2 & I had to play about half a beat ahead of [the pianist] to hold the time but it was so far ahead that I found it difficult to comp in any meaningful way or to hold the standard swing groove particularly well \\ 
Piano & 3 & 2 & Tried to lead tempo again but gave up - starting to think he is taking cues from the content of what I am playing? Seems to follow a similar pattern of slowing down nonetheless \\ 
Piano & 4 & 1 & Similar creative energy. Seemed to be similar disruptions through the feed, but these helped with the flow of the music. Several moments of synchronising with each other during a fill. \\ 
Piano & 4 & 2 & Sunk into quite a deep groove. Subtle disruptions in the feed were subsumed within the strength of our interaction/groove. \\ \bottomrule
\end{tabular}
\end{sidewaystable}\clearpage

\begin{sidewaystable}[]
\centering
\caption{Latency: 180ms, Jitter: 0.0x}
\begin{tabular}{l c c p{15cm}}
\toprule
\textbf{Instrument} & \textbf{Duo} & \textbf{Session} & \textbf{Comment} \\ \midrule
Drums & 1 & 1 & Tried to just play right tempo at the beginning. Took a while but we settled into a fairly stable groove. Pretty alright \\ 
Piano & 1 & 1 & Everything seemed to align this time, including no confused feedback from [the drummer]. I felt a bit tentative though that maybe I was actually just way off without realising it! \\ 
Drums & 1 & 2 & I planned to stick to my guns but it's always really difficult. Regardless I feel that one went well, and we even synced up for some pushes etc \\ 
Piano & 1 & 2 & Felt mostly ok and even interactive between feels and fills .. something seemed a bit off but might have just been doubt that we were actually sounding ok! \\ 
Drums & 2 & 1 & Being out by two quaver triplets (or one the other way) yielded some good musical results. \\ 
Drums & 3 & 1 & We rushed a little bit but managed to play mostly together after about 30 seconds albeit at a faster tempo \\ 
Piano & 3 & 1 & [The drummer] started faster and I tried to adjust to him - became more of a shuffle feel. Once we locked in it felt there was most interaction in this test. \\ 
Drums & 3 & 2 & [The pianist] came in several seconds after me so I just held the tempo and we managed to play together for the whole 90 seconds however I got lost in the form pretty quickly so my accents were a bit random \\ 
Piano & 3 & 2 & We were on the same page with this one - clearly feeling a similar tempo and both able to play more freely \\ 
Piano & 4 & 1 & Started normal, then switched to the previous Condition 5. However, we were even more interactive at accommodating. Worked out at essentially a blues in 7/4! \\ 
Piano & 4 & 2 & Again, we were a beat out. Got so used to it that we were constantly trying to find the tempo from each other – surpassed falling into a 7/4 time signature, and instead became changes every few crotchets! \\ 
Drums & 5 & 1 & I got lost and then flipped the beat so that I was with [the drummer] by the end \\ \bottomrule
\end{tabular}
\end{sidewaystable}\clearpage

\begin{sidewaystable}[]
\centering
\caption{Latency: 180ms, Jitter: 0.5x}
\begin{tabular}{l c c p{15cm}}
\toprule
\textbf{Instrument} & \textbf{Duo} & \textbf{Session} & \textbf{Comment} \\ \midrule
Drums & 1 & 1 & Did a fill, we went forward at different speeds, was displaced by a beat. Got it back but bit great. Feel the beginning was pretty good actually \\ 
Piano & 1 & 1 & Again I seemed to be having an easier time of it than [the drummer]. From my perspective the performance was most effective when I ignored what [the drummer] was doing and focused on my own playing in relation to what I could hear. Even when looking at the screen I think I was ignoring what I could see in a way \\ 
Drums & 1 & 2 & Tempo stayed the same pretty much. We were mostly in sync. Did a bit of looking to each other for pushes etc. Didn't feel bad \\ 
Piano & 1 & 2 & I felt like things were unravelling but [the drummer] seemed quite happy on the screen so I wonder if there was the same disconnect that I commented on earlier but in reverse \\ 
Drums & 2 & 1 & This was more like being on parallel tracks rather than being locked in together. \\ 
Drums & 3 & 1 & We had to coordinate our beats at the start but then managed to play together for the rest of the take, but there was a tendency to rush \\ 
Piano & 3 & 1 & Hi hat moved about between 1 and 3 and 2 and 4. We agreed on tempo after a while but never fully settled because of the beat switching \\ 
Drums & 3 & 2 & We had to coordinate at the start because we pretty quickly dropped a beat but then it was fine. We rushed a bit but it felt like we were playing together and could communicate clearly \\ 
Piano & 3 & 2 & Tempo felt quite secure - hi hat flipped again - I added an extra beat towards the end of the recording to get back on 2 and 4 \\ 
Piano & 4 & 1 & I believe that [the drummer] and I were a crotchet out from each other! However, we soon realised this, and were able to attempt to adjust to the other repeatedly (every 2 bars or so) – therefore quite interactive! \\ 
Piano & 4 & 2 & Most of performance was a beat out from [the drummer]. We were interacting a lot, implying the tempo in ambiguous ways deliberately, to put the other off! \\ \bottomrule
\end{tabular}
\end{sidewaystable}\clearpage

\begin{sidewaystable}[]
\centering
\caption{Latency: 180ms, Jitter: 1.0x}
\begin{tabular}{l c c p{15cm}}
\toprule
\textbf{Instrument} & \textbf{Duo} & \textbf{Session} & \textbf{Comment} \\ \midrule
Drums & 1 & 1 & Difficult to decide whether to plow on at correct speed when things go awry or to try and match keys \\ 
Piano & 1 & 1 & Part of me thinks I was just overthinking the experiment but there were a couple of points where our communication really broke down and the performance suffered. I put 7 for interactivity because we still responded quickly using the visuals, expressing the failure \\ 
Piano & 1 & 2 & For the most part we felt interactive and that our feedback was aligned \\ 
Drums & 2 & 1 & My vocabulary shrinks and becomes much more 'pragmatic' when the time isn't settled. \\ 
Drums & 3 & 1 & At the beginning it took us a second to figure out how to play in time, and once we came together there were several points where we were suddenly playing on different beats to each other but it was easy to add/drop a beat to come back in time and then it felt like we were together. It was easier to listen than watch on this one \\ 
Piano & 3 & 1 & Tempo quickly increased - I noticed the hi hat flip to 1 and 3 in the process and tried to correct it by shifting a beat - we then seemed to flip back at some point. I was more thinking about the tempo so didn’t attempt to sync up again - even though the tempo was quite solid we were not always arriving at beat 1 together \\ 
Drums & 3 & 2 & I had to play a beat ahead of [the pianist] and once I realised that it was mostly ok but felt quite mechanical. I got confused with the phrasing once and dropped a beat but [the pianist] realised and we switched back to the previous arrangement \\ 
Piano & 3 & 2 & Felt like [the drummer] was following my tempo this time - there was some push and pull at the beginning - we seemed to skip a beat somewhere at the beginning and the hi hat flipped once \\ 
Piano & 4 & 1 & Perhaps the most tumultuous feed – lots of disruptions/speed differences/changes of beat. But having become accustomed to all of these, we were able to use them to interact consistently and vibrantly, hopefully generating lots of creative energy. \\ 
Piano & 4 & 2 & Lots of feed disruption. Very interactive though! \\ 
Piano & 5 & 1 & If either of us get out of sync the other is quick to readjust \\ 
Piano & 5 & 2 & We came in at different times \\ \bottomrule
\end{tabular}
\end{sidewaystable}