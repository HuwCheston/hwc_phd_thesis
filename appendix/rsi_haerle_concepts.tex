\chapter{Harmonic Concepts}\label{chap:rsi_haerle_concepts}

The following table describes the twenty harmonic concepts taken from \citep{Haerle1994} that are used in Section~\ref{sec:rsi_factorised_cavs}.\clearpage

\begin{sidewaystable}[ht]
    \centering
    \caption[Descriptions of the twenty chapters in \citet{Haerle1994} that make up the concept dataset.]{Descriptions of the twenty chapters in \citet{Haerle1994} that make up the concept dataset. The text is adapted and summarised from the didactic descriptions that precede each chapter.}
    \label{tab:rsi_sm_haerle_concepts}
    \begin{tabular}{lcp{15cm}}
        \toprule
         \textbf{Name} 
&\textbf{Number} & \textbf{Description} \\
        \midrule
         Block Chords 
&1 & Includes ``block''-type, closed chords that are built on stacked thirds and outline common voicings for major, dominant, and minor, and diminished sevenths, as well as the same chords with suspended fourths. \\
         Shell Voicings 
&2 & Includes ``shell''-type, open chord voicings that outline extensions beyond the seventh (e.g., sixths, ninths, flattened fifths). \\
         Diatonic 7th Chords 
&3 & Includes diatonic seventh chords walked sequentially through major and minor tonalities. \\
         Cycle Progressions 
&4 & Connects different inversions of voicings in an idiomatic way as part of typical progressions. Common progressions represented in this concept are I-IV in major and minor, dominant 7th cycles, I-V in major and V-I in major and minor. \\
         II-V-I's in Major and Minor 
&5 & A logical extension of the Cycle Progressions concept which moves on to playing complete II-V-I cadences in major and minor keys. \\
         I-IV Cycle Progression 
&6 & A common cycle progression which moves from the key centre to the IV chord which, in turn, becomes a new key centre. \\
         Modal Fourthy Voicings 
&7 & Fourthy voicings which are moved through a Dorian mode. \\
         ``So What'' Voicings 
&8 & A particular fourthy structure probably best known from its use in the Miles Davis composition ``So What''. Presented in both minor (Dorian) and major (Lydian) modes. \\
         Modal ``So What'' Voicings 
&9 & Fourthy voicings which are moved through a Dorian mode. The ``So What'' voicing is used and, rather than keeping its pure structure, it changes slightly as it is moved through the Dorian scale. \\
         Fourthy II-V-I's 
&10 & II-V-I progressions using the ``So What'' voicing. The structure is built on the root of the II chord and on the 3rd of the I chord. Since the voicing doesn't clearly fit or imply a dominant chord, the V chord simply involves chromatin side-slipping up or down to the I chord. \\
    \end{tabular}
\end{sidewaystable}

\begin{sidewaystable}[ht]
    \centering
    \begin{tabular}{lcp{15cm}}
             Tri-Tone Sub II-V-I's 
&11 & II-V-I progressions using both the normal II and V chords and the II-V located a tri-tone away. The result is a deception as though the cadence was suddenly modulating to a distant key. \\
         Polychordal II-V-I's 
&12 & II-V-I progressions involving polychordal structures in which the left hand plays a conventional inversion and the right hand plays some kind of triadic structure to create extensions and/or alterations of the harmony. \\
         Cycling Altered Dominants 
&13 & Progressions involving polychordal structures in which the left hand plays a dominant structure and the right hand plays some kind of triadic structure to create extensions and/or alterations of the harmony. \\
         Polychordal Blues Voicings 
&14 & Twelve-bar blues progressions involving polychordal structures in which the left hand plays a conventional inversion and the right hand plays some kind of triadic structure to fill out a two-hand voicing of the harmony. \\
         Fourthy Blues Voicings 
&15 & Blues progressions involving fourthy structures in which the left hand plays a conventional inversion and the right hand plays a structure of two perfect fourths to fill out a two-hand voicing of the harmony. \\
         Major 7th Blues Voicings 
&16 & Blues progressions incorporating voicings from other concepts, including II-V-I's in Major and Minor (concept 5), I-IV Cycle Progression (6), and Tri-Tone Sub II-V-I's (11). \\
         Minor Blues Voicings 
&17 & Minor blues progressions involving fourthy structures in which the left hand plays a conventional inversion and the right hand plays a structure of two perfect fourths to fill out a two-hand voicing of the harmony. \\
         Dominant 7th Polychords 
&18 & Dominant seventh chords where the left hand plays a conventional inversion and the right hand plays some additional construction (e.g., involving flattening and sharpening of the fifth and ninth scale degrees). \\
         Dominant Polychord Groups 
&19 & Utilizes five of the polychord formulas from the previous concept (Dominant 7th Polychords) in groups with each other against a conventional inversion in the left hand to create melodic motion. \\
         Diminished Substitutions &20 & Utilizes voicings that are derived from the half-whole diminished scale that relates to a dominant seventh chord. Since the same scale relates to four different dominant seventh chords, it makes this device possible. \\
        \bottomrule
    \end{tabular}
\end{sidewaystable}